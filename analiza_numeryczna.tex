\documentclass[12pt]{article}
\usepackage[T1]{fontenc}
\usepackage[utf8]{inputenc}
\usepackage{polski}

\usepackage{amsmath}
\usepackage{color}

\begin{document}

\begin{table}[h!]
\caption{Porównanie wyestymowanej i 'prawdziwej' ceny standardowej opcji amerykańskiej sprzedaży dla różnych cen wykonania $K$.}
\centering
\begin{tabular}{ccccc}
\hline                       
 & \textbf{Dolny estymator} & \textbf{Górny estymator} & \textbf{Cena} & \textbf{Cena}\\
\textbf{K} & \textbf{$\Phi$} & \textbf{$\Theta$} & \textbf{estymowana} & \textbf{rzeczywista}\\
\hline
$85$&1.384&1.401&1.3926251316453448&1.4153195143471118\\
$90$&2.428&2.460&2.444366863759891&2.4730582818844775\\
$95$&3.926&3.989&3.957720496593192&4.026193509209022\\
$100$&5.969&6.055&6.012169206334639&6.1097356254733155\\
$105$&8.553&8.677&8.614906709930144&8.751758478196695\\
$110$&11.648&11.836&11.741864954859267&11.973945305957859\\
$115$&15.146&15.616&15.380842006119906&15.773578169346239\\
\hline 
\end{tabular}
Pozostałe parametry: $S_0 = 100$, $r=0.05$, $\sigma = 0.2$, $T=1$ oraz $b = 100$,$N = 100$. Czasy potencjalnego wykonania to: $0,\frac{T}{3},\frac{2T}{3}$ i $T$.
\label{tab:wycena_am} 
\end{table}

\begin{table}[h!]
\caption{Porównanie wyestymowanej i 'prawdziwej' ceny standardowej opcji amerykańskiej sprzedaży dla różnych ilości gałęzi $b$.}
\centering
\begin{tabular}{ccccc}
\hline                       
 & \textbf{Dolny estymator} & \textbf{Górny estymator} & \textbf{Cena} & \textbf{Cena}\\
\textbf{b} & \textbf{$\Phi$} & \textbf{$\Theta$} & \textbf{estymowana} & \textbf{rzeczywista}\\
\hline
25&3.909&4.097&4.003002295619451&4.026193509209022\\
50&3.771&3.869&3.8197018442634576&4.026193509209022\\
75&3.888&3.958&3.9234171510972713&4.026193509209022\\
100&3.926&3.989&3.957720496593192&4.026193509209022\\
125&3.854&3.896&3.8753735798349487&4.026193509209022\\
\hline 
\end{tabular}
Pozostałe parametry: $S_0 = 100$, $r=0.05$, $\sigma = 0.2$, $T=1$, $K = 95$ oraz $N = 100$. Czasy potencjalnego wykonania to: $0,\frac{T}{3},\frac{2T}{3}$ i $T$.
\label{tab:wycena_am_b} 
\end{table}


\begin{table}[h!]
\caption{Dane dla Facebook na dzień 07.06.2019}
\centering
\begin{tabular}{|c|c|}
\hline                       
\textbf{Parametr} & \textbf{Wartość} \\
\hline
$S_0$&168.33\\
K&145\\
$\sigma$&15.50\%\\ %wzięte VIX volatility 
r&2.25\%\\ %amerykańska interest rate
\hline 
\end{tabular}
\label{tab:wycena_fb} 
\end{table}

\begin{table}[h!]
\caption{Dane dla Apple na dzień 07.06.2019}
\centering
\begin{tabular}{|c|c|}
\hline                       
\textbf{Parametr} & \textbf{Wartość} \\
\hline
$S_0$&185.22\\
K&190\\ %przykladowe
$\sigma$&15.50\%\\ %wzięte VIX volatility 
r&2.25\%\\ %amerykańska interest rate
\hline 
\end{tabular}
\label{tab:wycena_apple} 
\end{table}
\end{document}